\chapter{Discrete Random Variables and Expectation}

\section{Comments on the main text}

\section{Exercises}

\subsection*{Exercise 2.1}

(k+1)/2

\subsection*{Exercise 2.2}

The probability that typing ``proof'' is $\frac{1}{26^5}$. Let $X_i$ be the random variable
that is 1 if the word ``proof'' starts at the position $i$. Therefore, by linearity of 
expectation

\begin{equation*}
\E[X] = \sum_{i=1}^{999996} \E[X_i] = \frac{999996}{26^5}
\end{equation*}

\subsection*{Exercise 2.3}

By Jensen's inequality, if $f$ is convex, then we know that $\E[f(X)] \geq f(\E[X])$.
If it is a linear function then we know that $\E[f(X)] = f(\E[X])$. If it is concave
then $\E[f(X)] \leq f(\E[X])$.

\subsection*{Exercise 2.4}

By Jensen's inequality, $\E[f(X)] \geq f(\E[X])$ for any convex functions $f$. If
$f$ is twice differentiable and its second derivative is non-negative, then $f$ is convex. For
$f(x) = x^k$, the second derivative is $f''(x) = k(k-1)x^{k-2}$. Since $k$ is an even number greater
than 1, we know that the second derivative is non-negative. Hence, it is convex. 

\subsection*{Exercise 2.5}

The probability that $X$ is even is that

\begin{equation*}
\begin{split}
\P[X \text{ is even}] & = \sum_{i=0,2,4...}^{n} \binom{n}{i}\left(\frac{1}{2}\right)^{n} \\
& = \left(\frac{1}{2}\right)^{n} \sum_{i=0,2,4...}^{n} \binom{n}{i}
\end{split}
\end{equation*}

We know that

\begin{equation*}
\begin{split}
\binom{n}{0} & = \binom{n-1}{0} \\
\binom{n}{1} & = \binom{n-1}{1} + \binom{n-1}{0} \\
\binom{n}{2} & = \binom{n-1}{2} + \binom{n-1}{1} \\
\binom{n}{3} & = \binom{n-1}{3} + \binom{n-1}{2} \\
... \\
\binom{n}{n-2} & = \binom{n-1}{n-2} + \binom{n-1}{n-3} \\
\binom{n}{n-1} & = \binom{n-1}{n-1} + \binom{n-1}{n-2} \\
\binom{n}{n} & = \binom{n-1}{n-1} \\
\end{split}
\end{equation*}

We found that when $n$ is even

\begin{equation*}
\begin{split}
\sum_{i=0,2,4...}^{n} \binom{n}{i} & = \binom{n}{0} + \binom{n}{2} + \binom{n}{4} + ... + \binom{n}{n} \\
& = \binom{n-1}{0} + \binom{n-1}{1} + \binom{n-1}{2} + ... + \binom{n-1}{n-1} \\
& = \binom{n}{1} + \binom{n}{3} + \binom{n}{5} + ... + \binom{n}{n - 1}
\end{split}
\end{equation*}

and when $n$ is odd

\begin{equation*}
\begin{split}
\sum_{i=0,2,4...}^{n} \binom{n}{i} & = \binom{n}{0} + \binom{n}{2} + \binom{n}{4} + ... + \binom{n}{n-1} \\
& = \binom{n-1}{0} + \binom{n-1}{1} + \binom{n-1}{2} + ... + \binom{n-1}{n-1} \\
& = \binom{n}{1} + \binom{n}{3} + \binom{n}{5} + ... + \binom{n}{n}
\end{split}
\end{equation*}

Hence, $\sum_{i=0,2,4...}^{n} \binom{n}{i} = 2^{n-1}$. $\P[X \text{ is even}] = 1/2$.

\subsection*{Exercise 2.6}

Two six-sided dice. $X = X_1 + X_2$.

(a) $\E[X | X_1 \text{ is even}] = \frac{3}{18} + \frac{4}{18} + \frac{10}{18} + \frac{12}{18} + \frac{21}{18} + \frac{24}{18}
+ \frac{18}{18} + \frac{20}{18} + \frac{11}{18} + \frac{12}{18}$.

(b) $\E[X | X_1 = X_2] = 7$.

(c) $\E[X_1 | X = 9] = 9/2$.

(d) $\E[X_1 - X_2 | X = k] = 0 $ for $k \in [2,12]$.


\subsection*{Exercise 2.7}

$X$ and $Y$ are independent geometric random variables, where $X$ has parameter $p$ and $Y$ has parameter $q$.

(a) What is $\P[X = Y]$?

$\P[X = Y = n] = (1-p)^{n-1}p(1-q)^{n-1}q$. 

$\P[X=Y] = \sum_{n=1}^{\infty} (1-p)^{n-1}p(1-q)^{n-1}q
= \sum_{n=1}^{\infty} pq[(1-p)(1-q)]^{n-1}$.

Since we know that for geometric random variables 
$\P[X \geq i] = \sum_{n=i}^{\infty}(1-p)^{n-1}p = (1-p)^{i-1}$. So we have,

\begin{equation*}
\begin{split}
S = \P[X = Y] & = \sum_{n=1}^{\infty} pq[(1-p)(1-q)]^{n-1} \\
\frac{(p+q-pq)S}{pq} & = \sum_{n=1}^{\infty}(1-p-q+pq)^{n-1}(p+q-pq)\\
& = (1-p-q+pq)^{1-1} = 1\\
S & = \frac{pq}{p+q-pq}
\end{split}
\end{equation*}

Hence, $\P[X = Y] = \frac{pq}{p+q-pq}$.

(b) What is $\E[\max(X,Y)]$?

From \textbf{MU 2.9} we know that $\E[\max (X,Y)] = \E[X] + \E[Y] - \E[\min(X,Y)]$.
From part (c) we will know that $\min(X,Y)$ is a geometric random variable mean $p+q-pq$. 
Therefore, $\E[\min(X,Y)] = \frac{1}{p+q-pq}$, and we get

\begin{equation*}
	\E[\max(X,Y)] = \frac{1}{p} + \frac{1}{q} - \frac{1}{p+q-pq}
\end{equation*}

(c) What is $\P[\min(X,Y) = k]$?

We split this event into two disjoint events.

\begin{equation*}
\begin{split}
\P[\min(X,Y) = k] & = \P[X = k, Y \geq k] + \P[X > k, Y = k] \\
& = \P[X = k] \cdot \P[Y \geq k] + \P[X > k] \cdot \P[Y = k] \\
& = (1-p)^{k-1}p \cdot \sum_{n=k}^{\infty}(1-q)^{n-1}q + \left(\sum_{m=k+1}^{\infty}(1-p)^{m-1}p\right) \cdot (1-q)^{k-1}q \\
& = (1-p)^{k-1}p \cdot (1-q)^{k-1} + (1-p)^{k} \cdot (1-q)^{k-1}q \\
& = (1-p)^{k-1}(1-q)^{k-1}(p + (1-p)q) \\
& = [(1 - (p+q-pq)]^{k-1}(p+q-pq)
\end{split}
\end{equation*}

(d) What is $\E[X | X \leq Y]$?

\begin{equation*}
\begin{split}
\E[X | X \leq Y] & = \sum_{x} x \frac{\P[X = x \cap x \leq Y]}{\P[X \leq Y]}\\
\end{split}
\end{equation*}

We investigate our denominator,

\begin{equation*}
\begin{split}
\P[X \leq Y] & = \sum_{x = 1}^{\infty}\P[Y \geq x] \P[X = x]\\
& = \sum_{x = 1}^{\infty}(1-q)^{x-1} (1-p)^{x-1}p \\
& = p \sum_{x=1}^{\infty}(1-p-q+pq)^{x-1} \\
& = \frac{p}{p+q-pq}
\end{split}
\end{equation*}

Now the whole equation is

\begin{equation*}
\begin{split}
\E[X | X \leq Y] & = \sum_{x} x \frac{\P[X = x \cap x \leq Y]}{\P[X \leq Y]}\\
& = \frac{p+q-pq}{p} \cdot \sum_x x \P[X = x \cap x \leq Y]\\
& = \frac{p+q-pq}{p} \cdot \sum_x x \P[X = x]\cdot \P[x \leq Y]\\
& = \frac{p+q-pq}{p} \cdot \sum_x x (1-p)^{x-1}p \cdot (1-q)^{x-1}\\
& = \sum_{x} x (1-p-q+pq)^{x-1}(p+q-pq)
\end{split}
\end{equation*}

This is equivalent to the expectation of a geometric random variable with mean $p + q - pq$.
Hence, $\E[X | X \leq Y] = \frac{1}{p+q-pq}$.

\subsection*{Exercise 2.8}

Let $G$ be the random variable that represents the number of girls they have and let $B$ be the number 
of girls they have.

(a) $\E[G] = 0 \cdot (\frac{1}{2})^k + 1 \cdot (\frac{1}{2} + \frac{1}{2^2} + \frac{1}{2^3} + ... + \frac{1}{2^{k}})
= \frac{2^{k}-1}{2 ^k}$.

$\E[B] = \sum_{n = 0}^{k - 1} \frac{n}{2^{n+1}} + \frac{k}{2^k} = \frac{2^k -1}{2^k}$.

(b) We know that the expectation of the number of their children is $2$ since now it is a standard
geometric random variable. The expected number of girls they can have is

\begin{equation*}
	\E[G] = \sum_{i = 1}^{\infty}\frac{1}{2^i} = 1
\end{equation*}

Therefore, $\E[B] = 1$.

\subsection*{Exercise 2.9}

(a) 

\begin{equation*}
\begin{split}
\E[\max(X_1,X_2)] & = \sum_{x_1}\sum_{x_2}\max (x_1, x_2)(1/k)(1/k) \\
& = 1/k^2 \sum_{x_1}\sum_{x_2 \leq x_1} x_1 + \sum_{x_2 > x_1}x_2
\end{split}
\end{equation*}

\begin{equation*}
\begin{split}
\E[\min(X_1,X_2)] & = \sum_{x_1}\sum_{x_2}\min (x_1, x_2)(1/k)(1/k) \\
& = 1/k^2 \sum_{x_1}\sum_{x_2 \leq x_1} x_2 + \sum_{x_2 > x_1}x_1
\end{split}
\end{equation*}

(b)

\begin{equation*}
\begin{split}
\E[\max(X_1,X_2)] + \E[\min(X_1,X_2)] & = 1/k^2 \sum_{x_1}\sum_{x_2 \leq x_1} x_1 + \sum_{x_2 > x_1}x_2 
+ 1/k^2 \sum_{x_1}\sum_{x_2 \leq x_1} x_2 + \sum_{x_2 > x_1}x_1\\
& = 1/k^2 \sum_{x_1} \left( \sum_{x_2}x_2 + \sum_{x_2} x_1 \right) \\
& = 1/k^2 \sum_{x_1}\sum_{x_2}(x_2 + x_1) \\
& = \E[X_1] + \E[X_2]
\end{split}
\end{equation*}

(c)

By linearity of expectation,

\begin{equation*}
\begin{split}
\E[\max(X_1,X_2)] + \E[\min(X_1,X_2)] & = \E[\max(X_1, X_2) + \min(X_1, X_2)] \\
& = \E[X_1 + X_2]\\
& = \E[X_1] + \E[X_2]
\end{split}
\end{equation*}


\subsection*{Exercise 2.10}

(a)

When $n = 1$, there is only one $\lambda = 1$.

\begin{equation*}
\begin{split}
f(\lambda x) \leq \lambda f(x)
\end{split}
\end{equation*}

Inductive step, assuming for $n = k$, $f(\sum_{i=1}^{k}\lambda_i x_i) \leq \sum_{i=1}^{k}\lambda_i f(x_i)$.
Then we have, 

\begin{equation*}
\begin{split}
f(\sum_{i=1}^{k+1}\lambda_i x_i) & = f(\sum_{i=1}^{k-1}\lambda_i x_i + \lambda_k x_k + \lambda_{k+1} x_{k+1})\\
& = f(\sum_{i=1}^{k-1}\lambda_i x_i + (\lambda_k + \lambda_{k+1})(\frac{\lambda_k x_k}{\lambda_k + \lambda_{k+1}} 
+ \frac{\lambda_{k+1} x_{k+1}}{\lambda_k + \lambda_{k+1}}))\\
& = f(\sum_{i=1}^{k-1}\lambda_i x_i + \lambda_k' x_k') \\
& \leq \sum_{i=1}^{k}\lambda_i f(x_i) \\
& = \sum_{i=1}^{k-1}\lambda_i f(x_i) + \lambda_{k}'f(x_k') \\
& = \sum_{i=1}^{k-1}\lambda_i f(x_i) + (\lambda_k + \lambda_{k+1})f(\frac{\lambda_k x_k}{\lambda_k + \lambda_{k+1}} 
+ \frac{\lambda_{k+1} x_{k+1}}{\lambda_k + \lambda_{k+1}}) \\
& \leq \sum_{i=1}^{k-1}\lambda_i f(x_i) + (\lambda_k + \lambda_{k+1})(\frac{\lambda_k}{\lambda_k + \lambda_{k+1}}f(x_k) 
+ \frac{\lambda_{k+1}}{\lambda_k + \lambda_{k+1}}f(x_{k+1})) \\
& = \sum_{i=1}^{k-1}\lambda_i f(x_i) + \lambda_k f(x_k) 
+ \lambda_{k+1} f(x_{k+1}) \\
& = \sum_{i = 1}^{k+1} \lambda_i f(x_i)
\end{split}
\end{equation*}

(b)

If $X$ can only take finitely many values, then we can write its probability distribution as $\sum_{i=1}^{n}\lambda_i = 1$
where $\lambda_i$ is the probability that $X$ takes the $i$th value in Im$X$.

\subsection*{Exercise 2.11}

Prove Lemma 2.6.

\subsection*{Exercise 2.12}

The expected number of cards we must draw to see all cards is

\begin{equation*}
\begin{split}
\E[X] = \sum \E[X_i] = n \sum_{i=1}^{n} \frac{1}{i}
\end{split}
\end{equation*}

If we draw $2n$ cards, what is the expected number of cards in the deck that are not chosen at all?

The idea is to use indicator random variables. 
The probability that the $i$th card is not chosen is $(\frac{n-1}{n})^{2n}$. 

\begin{equation*}
\begin{split}
\E[X] = \sum \E[X_i] = n (\frac{n-1}{n})^{2n}
\end{split}
\end{equation*}

Chosen exactly once?

The probability that the $i$th card is chosen only once is 
$\binom{2n}{1}(\frac{1}{n})(\frac{n-1}{n})^{2n-1}$.

\begin{equation*}
\begin{split}
\E[X] = \sum \E[X_i] = 
n \binom{2n}{1}(\frac{1}{n})(\frac{n-1}{n})^{2n-1}
= 2n (\frac{n-1}{n})^{2n-1}
\end{split}
\end{equation*}

\subsection*{Exercise 2.13}

(a)

This problem is exactly equivalent to the coupon collector’s problem we did; assume that the
relevant pairs of coupons form a label. 

(b) No matter what $k$ is, we can always get $n\ln n + \Theta(n)$.

\subsection*{Exercise 2.14}

The last toss must be head, so for the previous $n-1$ tosses we know there are $k-1$ heads.
Hence, the probability is $\binom{n-1}{k-1} p^k(1-p)^{n-k}$.

\subsection*{Exercise 2.15}

Let $X_i$ be the number of coin flips for the next head. 

\begin{equation*}
\begin{split}
\E[X] & = \E[\sum_{i = 1}^{k} X_i] \\
& = \sum_{i=1}^{k} \E[X_i] \\
& = \sum_{i=1}^{k} 1/p = k/p
\end{split}
\end{equation*}

\subsection*{Exercise 2.16}

(a)

Let $X_i$ be an indicator random variable which gets 1 if there is a streak of $\log n + 1$ starting
from the $i$th flip. Its expectation is $(1/2)^{\log n} = 1/n$ because if a streak starts from the
$i$th flip, the next $\log n$ flips must be the same as the $i$th one. Hence the expectation of the number
of streaks of length $\log n + 1$ is $(n - \log n)(1/n) = 1 - o(1)$.


\noindent (b)

In other words, we want to prove that with high probability there is at least one streak with length
at least $\floor{\log n - 2\log \log n}$.

Let $k = \floor{\log n - 2\log \log n}$. We break the sequence into disjoint blocks of $k$ consecutive
flips. There are $\floor{n/k}$ such blocks. For the sequence of $n$ flips to not contain a streak of
$k$ flips (denote this event by $A$) it is necessary that none of the blocks contains a streak of length
$k$ (denote this event by $B$). Thus we have $\P[A] \leq \P[B]$. 

The probability that a single block does not contain a streak is $1 - (1/2)^{k-1}$. Since the blocks
are disjoint and independent, the probability that none of the blocks contains a streak is

\begin{equation*}
	\P[B] = \left(1 - \left(\frac{1}{2}\right)^{k-1} \right)^{\floor{n/k}}
\end{equation*}

Now, since $k - 1 \leq \log n - 2 \log n \log n$ and $\floor{n/k} \leq n/\log n - 1$, we get

\begin{equation*}
\begin{split}
\P[B] & \leq \left(1 - \left(\frac{1}{2}\right)^{\log n - 2\log \log n}  \right)^{n/\log n - 1} \\
& = \left(1 - \frac{\log^2 n}{n}\right)^{n/\log n - 1} \\
& \leq \left(\exp\left(- \frac{\log^2 n}{n}\right)\right)^{n/\log n - 1} \\
& = \exp\left(-\frac{\log^2 n}{n}\left(\frac{n}{\log n} - 1 \right)\right) \\
& = \exp\left(-\log n \left(1 - \frac{\log n}{n}\right)\right)
\end{split}
\end{equation*}
where the second inequality is based on the fact that $1 - x \leq e^{-x}$. Let $g(n) = 1 - \log n/n$.
Since $g(4) = 3/4 > 1/\log e$ and since $g(n)$ is increasing for $n \geq 4$ as the derivative $g'(n) = 
n^{-2}(\log n - 1/(\ln 2)) > 0$ when $n \geq 4$, for $n \geq 4$ we have that $g(n) \geq 1/\log e$ and 
therefore

\begin{equation*}
	\P[B] \leq \exp \left( - \frac{\log n}{\log e} \right) = \exp(-\ln n) = \frac{1}{n}
\end{equation*}

Note that here we do not try to find out the exact probability of the event described in the problem($A$).
Instead, we find another event $B$ which has higher probability than $A$. The probability of $B$ is 
easier to bound.  

The answer can be also found in \url{https://www.cs.helsinki.fi/u/mkhkoivi/teaching/RA-I/solutions1.pdf}

\subsection*{Exercise 2.17}

We see that $\E[Y_0] = 1$ and $\E[Y_1] = 2p$. Next for $i \geq 1$, we have
\begin{equation*}
	\E[Y_i | Y_{i-1} = j] = 2pj
\end{equation*}
so that

\begin{equation*}
	\E[Y_i] = \E[\E[Y_i | Y_{i-1}]] = 
	\sum_{j} \P[Y_{i-1} = j] 2pj = 2p \E[Y_{i-1}]
\end{equation*}

Thus, we have $\E[Y_i] = (2p)^i$. When $p < 1/2$, this probability is bounded.

\subsection*{Exercise 2.18}

Use Induction. 

Let $b_1, b_2,...,b_n$ be the values of the items observed at time $b_t$. We will prove this by
induction. Let $M_t$ be a random variable that takes the value of the item in memory at time $t$.
We need to show that at time $t$, $\P[M_t = b_i] = 1/t$ for all $1 \leq i \leq t$.

The base case is when $t = 1$, which is trivially true since $M_t = b_1$ with probability 1. Assume
that at time $t$, $\P[M_t = b_i] = 1/t$ for all $1 \leq i \leq t$. Now we prove that this
property holds for time $t + 1$. At time $t+1$, we set $M_{t+1} = b_{t+1}$ with probability $1/(t+1)$.
Therefore, $\P[M_{t+1} = b_{t+1}] = 1/(t+1)$. For the rest, $1 \leq i \leq t$,

\begin{equation*}
\begin{split}
\P[M_{t+1} = b_i] & = \P[\text{no swat at time }t \text{ and } M_t = b_i] \\
& = \P[\text{no swap at time } t] \P[M_t = b_i] \\
& = \frac{t}{t+1} \frac{1}{t} \\
& = \frac{1}{t+1}
\end{split}
\end{equation*}

The answer can be found at \url{https://inst.eecs.berkeley.edu/~cs174/fa10/sol2.pdf}.

\subsection*{Exercise 2.19}

If in 2.18, when the $k$th item appears, it replaces the item in memory with probability $1/2$.
Describe the distribution of the item in memory.

In this version of the algorithm, let $X$ be random variable which is equal to $i$ if and only if the item
in memory is the $i$th item.

Then the probability distribution should be: for $i = 2,...,n$. $\P[X = i] = (\frac{1}{2})^{n-i+1}$.
For the first item, the probability is $\P[X = 1] = (\frac{1}{2})^{n-1}$.

\subsection*{Exercise 2.20}

There are $n$ positions, each position can either be a fixed point or not.

Let $X_i$ be an indicator showing whether the $i$th position is a fixed point.
The permutation may put $i$ still at $i$ with probability $1/n$ since it has
$n$ choices. So we have $\sum_{i=1}^{n}\E[X_i] = n \cdot 1/n = 1$.

The solution can also be found at \url{http://www.cs.nthu.edu.tw/~wkhon/random/assignment/assign1ans.pdf}.

\subsection*{Exercise 2.21}

\begin{equation*}
\begin{split}
\E[\sum_{i=1}^{n} |a_i -i|] & = \sum_{i=1}^{n} \E[|a_i - i|] \\
& = \sum_{i=1}^{n} \sum_{j=1}^{n} \frac{1}{n} |j-i| \\
& = \sum_{i=1}^{n} \frac{1}{n} (\sum_{p = 1}^{i-1} p + \sum_{q=1}^{n-i}q) \\
& = \sum_{i=1}^{n} \frac{2i^2 + n^2 + n - 2i - 2ni}{2n} \\
& = \sum_{i=1}^{n} (\frac{i^2-i}{n} + \frac{n + 1}{2} - i) \\
& = \sum_{i=1}^{n} \frac{i^2 - i}{n} \\
& = \frac{n^2 - 1}{3}
\end{split}
\end{equation*}

\subsection*{Exercise 2.22}

If $i < j$ and $a_i > a_j$, let $Y_{ij}$ be an indicator random variable.
Let $Y = \sum_{i<j} Y_{ij}$. What we want is just the expectation of $Y$.

\begin{equation*}
\begin{split}
\E[Y] & = \sum_{i<j} \E[X_{ij}] \\
& = \sum_{i<j} \P[a_i > a_ j] \\
& = \sum_{i<j} \frac{1}{2} \\
& = \frac{(n-1)n}{4}
\end{split}
\end{equation*}

A similar solution can be seen at \url{http://www.cs.ox.ac.uk/people/varun.kanade/teaching/CS174-Fall2012/HW/HW2_sol.pdf}.

\subsection*{Exercise 2.23}

Let $X_i$ be the number of swaps needed when the $i$th element is inserted.

We know that $\E[X_i] = \frac{i - 1}{2}$.

\begin{equation*}
\begin{split}
\E[X] & = \sum_{i=1}^{n} \E[X_k] \\
& = \sum_{i=1}^{n} \frac{i - 1}{2} \\
& = \frac{n^2-n}{4}
\end{split}
\end{equation*}

\subsection*{Exercise 2.24}

We use the memoryless property. This is a very important trick. 
There are two cases for us to discuss: 
\begin{enumerate}
	\item if the first is not 6, then we forget it
	and start from the next. The expected number of flips should be $1 + \E[X]$.
	The corresponding probability for this case is $5/6$.
	\item if the first is 6, then we look at the second one. If it is not 6, then we forget it
	and keep going. If it is 6, we count it as 2.
\end{enumerate}

Let $A_i$ means the event that $i$th flip is six. $\P[A_i] = 1/6$.

\begin{equation*}
\begin{split}
\E[X] & = \P[\overline{A_1}]\E[X | \overline{A_1}]
+ \P[A_1]\E[X|A_1] \\
& = \P[\overline{A_1}]\E[X | \overline{A_1}] + 
\P[A_1](\P[\overline{A_2} | A_1] \E[X|A_1,\overline{A_2}] +  
\P[A_2 | A_1] \E[X|A_1,A_2]) \\
& = \P[\overline{A_1}]\E[X | \overline{A_1}] + 
\P[A_1](\P[\overline{A_2}] \E[X|A_1,\overline{A_2}] +  
\P[A_2] \E[X|A_1,A_2]) \\
& = \frac{5}{6}\E[X | \overline{A_1}] + 
\frac{1}{6}(\frac{5}{6} \E[X|A_1,\overline{A_2}] +  
\frac{1}{6} \E[X|A_1,A_2]) \\
\end{split}
\end{equation*}

Hence we know that,

$\E[X] = \frac{5}{6}(1 + \E[X]) + \frac{1}{6}(\frac{5}{6}(2+\E[X]) + \frac{1}{6}\cdot 2)
= \frac{35}{36}\E[X] + \frac{42}{36}$.

\subsection*{Exercise 2.25}

\subsection*{Exercise 2.26}

These directed arcs will generate disjoint cycles because a node can only be an end point for two
edges. 

Let $X_i^k$ be an indicator variable which is 1 if vertex $i$ belongs to a cycle of length $k$.
Let $Y_k = \sum_{i=1}^{n} X_n^k$ be the total number of nodes belonging to a $k$-cycle and let
$N_k$ be the number of $k$-cycles in the graph. By definition we must have $N_k = Y_k/k$
because these cycles are disjoint. Finally
let $N$ be the total number of cycles in the graph, which is $N = \sum_{k=1}^{n}N_k$.

We want to know how many permutations $\pi$ there are such that $X_i^k = 1$. If vertex $i$ belongs
to a $k$-cycle, then the next vertex $j = \pi(i)$ can not be vertex $i$, vertex $\pi(j)$ can be neither
$i$ nor $j$, and so on, until the $k$th vertex is again $i$. Thus, we can choose the consecutive $k-1$
vertices on the same cycle in $(n-1)(n-2)...(n-k+1)$ ways. The remaining $n-k$ vertices can be in any
order in $\pi$, so the number of possibilities is $(n-k)(n-k-1)...2\cdot 1$. The total number of permutations
such that $X_i^k = 1$ therefore $(n-1)!$. As there are $n!$ permutations in total, the probablity of such event
is $\P[X_i^k = 1] = (n-1)!/n! = 1/n$.

By using the linearity of expectations twice, we can get the expected number of cycles.

\begin{equation*}
\E[N] = \sum_{k=1}^{n}\E[N_k] = \sum_{k=1}^{n}\frac{1}{k}\E[Y_k]
= \sum_{k=1}^{n}\frac{1}{k} \sum_{i=1}^{n}\E[X_i^k] = 
\sum_{k=1}^{n}\frac{1}{k} \sum_{i=1}^{n}\frac{1}{n} = \sum_{k=1}^{n}\frac{1}{k} = H(n)
\end{equation*}

\subsection*{Exercise 2.27}

\begin{equation*}
\begin{split}
\E[X] & = \sum_{x=1}^{\infty} x \P[X = x] \\ 
& = \sum_{x=1}^{\infty} \frac{6}{\pi^2} \frac{1}{x}
\end{split}
\end{equation*}

This sum diverges since $\sum \frac{1}{x}$ diverges, so the expectation is not finite.

\subsection*{Exercise 2.28}

\subsection*{Exercise 2.29}

\subsection*{Exercise 2.30}

\subsection*{Exercise 2.31}

\subsection*{Exercise 2.32}

(a)

$E_i$s are disjoint events, therefore $\P[E] = \sum_{i=1}^{n} \P[E_i]$. For
$i \leq m$, $\P[E_i] = 0$, since none of them are selected. Now, for $i>m$ two indepedent
events make up $E_i$.

\begin{equation*}
\begin{split}
\P[E_i] & = \P[\text{the } i \text{th candidate is the best}] \cdot 
\P[\text{the } i \text{th candidate is chosen}] \\
& = \frac{1}{n} \cdot \P[\text{best of the first } i-1 \text{ candidates is in the first }m \text{ candidates}]\\
& = \frac{1}{n} \frac{m}{i-1}
\end{split}
\end{equation*}

Now, putting this all together, we get

\begin{equation*}
\begin{split}
\P[E] = \sum_{i=m+1}^{n} \P[E_i] = \frac{m}{n}\sum_{i=m+1}^{n}\frac{1}{i-1}
\end{split}
\end{equation*}

(b)

Using Lemma 2.10 from the book,

\begin{equation*}
	\P[E] \geq \frac{m}{n}\int_{m+1}^{n+1}\frac{1}{x-1}dx = 
	\ln(x-1)|_{m+1}^{n+1} = \frac{m}{n}(\ln n - \ln m)
\end{equation*}

and

\begin{equation*}
\P[E] \leq \frac{m}{n}\int_{m}^{n}\frac{1}{x-1}dx = 
\ln(x-1)|_{m}^{n} = \frac{m}{n}(\ln (n-1) - \ln (m-1))
\end{equation*}


(c)

Since the bound from above is concave, we can take the derivative to find the $m$
so that the function is maximized. 

\begin{equation*}
\frac{\mathrm{d}}{\mathrm{d} m}\frac{m}{n}(\ln n - \ln m) = \frac{\ln n}{n} - \frac{\ln m}{n} + \frac{1}{n} = 0
\end{equation*}

Then we get $\ln m = \ln n - 1$, which is 

\begin{equation*}
m = e^{\ln n -1} = e^{\ln n}e^{-1} = ne^{-1} = \frac{n}{e}
\end{equation*}

Substituting this $m$ back into the bound from part (b), we get

\begin{equation*}
\P[E] \geq \frac{1}{e} (\ln n - \ln \frac{n}{e}) = 1/e
\end{equation*}































